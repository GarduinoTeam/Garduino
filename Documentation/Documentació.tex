\documentclass[11pt,a4paper]{article}
\usepackage[utf8]{inputenc}
%\usepackage[catalan]{babel}
\usepackage{comment}
\usepackage{adjustbox}
\usepackage{amsmath}
\usepackage{amsfonts}
\usepackage{amssymb}
\usepackage{graphicx}
\usepackage{fancyhdr}
\usepackage{appendix}
\usepackage{subfig}
\usepackage[ampersand]{easylist}
\usepackage{multirow}
\usepackage[hidelinks]{hyperref}
\usepackage[left=2cm,right=2cm,top=2cm,bottom=2cm]{geometry} 


\begin{document}
\begin{titlepage}

\begin{flushleft}
Escola Politècnica Superior\\
\vspace*{0.15in}
Master of Computer Engineering\\
\vspace*{0.15in}
ICT Project: Development and Implementation
\end{flushleft}

\begin{center}
\vspace{2.0cm}\includegraphics[scale=0.3]{figures/M-UdL.jpg}
\vspace{5.0cm}

\begin{LARGE}
\textbf{Sprint 1:}\\ 
\vspace*{0.15in}
\textbf{Requirement definition and system analysis}
\end{LARGE}
\vspace{5.0cm}

\vspace*{0.25in}
\rule{80mm}{0.1mm}\\
\vspace*{0.1in}

\begin{large}
\textbf{Students:}

\begin{tabular}{ll}
Adrià Casals Espax  & Joan Pau Castells Gasia \\
Gerard Donaire Alos  & Roger Truchero Visa \\
\multicolumn{2}{c}{David Sarrat González}
\end{tabular}
\\
\textbf{Date:} \today
\end{large}

\end{center}
\end{titlepage}

\lhead[\thepage]{\includegraphics[scale=0.05]{figures/M-UdL.jpg}  }
\chead[]{\textbf{Automatic irrigation system with plague detection}}
\rhead[]{ICT Project}
\renewcommand{\headrulewidth}{0.5pt}
\renewcommand{\footrulewidth}{0.5pt}
\fancypagestyle{plain}{
\fancyhead[L]{}
\fancyhead[C]{}
\fancyhead[R]{\thepage}
\fancyfoot[L]{}
\fancyfoot[C]{}
\fancyfoot[R]{}
\renewcommand{\headrulewidth}{0pt}
\renewcommand{\footrulewidth}{0pt}
}
\pagestyle{fancy}
\vspace*{0.05in}

\tableofcontents

\newpage

\vspace*{0.3in}
\listoftables
\listoffigures
\newpage

\part*{Introduction}
\addcontentsline{toc}{part}{Introduction}
	
The main idea is to design a watering system that allows us from a mobile app to water our plants remotely. This system will have different options of watering: Manual, planned and automatic sheet that the user can configure from the mobile app.\\

Additionally, we will install a plague recognition system formed by a neuronal convoluted network model that will feed images that you will receive from the webcam of the Raspberry Pi. All this information can be obtained with the option to monitor.\\

This means that we can go on vacation peacefully without having to worry about our plants and with the option to see their current status.

\part{Sprint 1}
\section{Requirements}
\subsection{Non-functional requirements}
\begin{enumerate}
\item \textbf{Product}
	\begin{enumerate}
	
	\item \textbf{Accessibility and Usability}
		\begin{enumerate}
		\item Regardless of the style of interaction, the interface has to be simple and intuitive, providing a high level of interactivity and usability. The tasks will be as visual as possible, since the application will be oriented to all types of age ranges. 
		\end{enumerate}
	\item \textbf{Concurrency}
		\begin{enumerate}
		\item  The app has to be multi-user, allowing the usage of several users simultaneously of the application.
		\end{enumerate}
		
	\item \textbf{Portability}
		\begin{enumerate}
		\item \textbf{Adaptability (multi-device):} It is required that the design is "responsive" in order to ensure proper display on multiple devices, such as tablets, smartphones...
		\end{enumerate}
		
	\item \textbf{Support}
		\begin{enumerate}
		\item The mobile app will be developed by the Android platform.
		\end{enumerate}
	\end{enumerate}

\item \textbf{External}
	\begin{enumerate}
	\item \textbf{Privacy} 
		\begin{enumerate}
		\item All data management has to conform to the requirements of the organic law of data protection in order to preserve privacy in the processing of personal data.
		\end{enumerate}
		
	\end{enumerate}

\begin{comment}
\item \textbf{Other restrictions}
	\begin{enumerate}
	\item ??
	\end{enumerate}
\end{comment}
\end{enumerate}

\subsection{Functional requirements}
\subsection{List of functionalities and features}
\textit{List of functionalities and features that all services (Android app, web service and Arduino) will accomplish}

\subsubsection{Android app}
\begin{itemize}
\item \textbf{The Android app will have to} allow communication with the web service.

\item \textbf{The Android app will have to} allow each user to log in.

\item \textbf{The Android app will have to} allow each user to configure/create an irrigation unit independently.

\item \textbf{The Android app will have to} show all irrigation units configured by the user.

\item \textbf{The Android app will have to} allow the manual watering option for each watering unit.

\item \textbf{The Android app will have to} allow planned irrigation of each irrigation unit.

\item \textbf{The Android app will have to} allow the option of monitoring the crop with data obtained from sensors and the webcam.

\item \textbf{The Android app will have to} allow the change of language.
\end{itemize}

\subsubsection{Web service}
\begin{itemize}
\item \textbf{The web service will have to} allow communication with the Android application and the Arduino devices.

\item \textbf{The web service will have to} allow the execution of a cron to make requests to the Arduino every X time so that a user can automatically receive notifications of his crop.

\item \textbf{The web service will have to} allow the execution of a machine learning script in charge of pest detection and return the data back to the Android application.

\item \textbf{The web service will have to} be able to determine when watering is optimal (if the automatic watering function has been used without programming) and send the request to the Arduino.

\item \textbf{The web service will have to} allow CRUD operations directly from the DB.
\end{itemize}

\subsubsection{Detection plague script}
\begin{itemize}
\item \textbf{The pest detection algorithm will have to} be able to perform leaf trimming, and then pass it on to the convolutional neural network.

\item \textbf{The pest detection algorithm will have to}, using an image of a leaf as an input, determine if this is a pest or not.
\end{itemize}

\subsubsection{Arduino}
\begin{itemize}
\item \textbf{The Arduino system will have to} allow communication with the web service.

\item \textbf{The Arduino system will have to} allow images to be sent in real time to the web service.

\item \textbf{The Arduino system will have to} allow the data from the humidity, temperature and soil sensors to be captured.

\end{itemize}

\newpage

\section{Main use cases}
\begin{figure}[hbtp]
\centering
\includegraphics[scale=0.3195,angle=90,origin=c]{figures/usecasediagram.png}
\caption{Application use cases diagram}
\end{figure}

\newpage

\section{General architecture}
\subsection{System architecture}
\begin{figure}[hbtp]
\centering
\includegraphics[scale=0.6]{figures/AppArchitecture.png}
\caption{System app architecture}
\end{figure}

The architecture of our application is composed by:
\begin{itemize}
\item \textbf{Android App}: Graphical interface, in charge of collecting the data entered by the user.
\item \textbf{Web Service}: Makes middleware of the application. It manages requests between the Android app and the Arduino/Raspberry pi, the database, ...
\item \textbf{Arduino/Raspberry pi}: Device that will be in charge of the irrigation of each environment (gardening, flowerpot, ...).
\item \textbf{Database}: Manage all the data of the application, user credentials, register all requests, device info, ...
\end{itemize}

\newpage

\subsection{Arduino/Raspberry pi architecture}
\vspace*{3cm}
\begin{figure}[hbtp]
\centering
\includegraphics[scale=0.6]{figures/ArduinoArch.png}
\caption{Arduino and rapsberry pi architecture}
\end{figure}

\newpage

\section{Data model}
%% Data model
\begin{figure}[hbtp]
\centering
\includegraphics[scale=0.5]{figures/ModelDeDades.png}
\caption{Data model UML}
\end{figure}
In this data model diagram we can see how our data is structured and related with the different datatables. We explain the target of the different tables:

\begin{itemize}
\item \textbf{User}: It stores all the data related to our users.

\item \textbf{Request}: It is used to store the requests of the users  to a specific device and to check if these requests have been completed. 

\item \textbf{Mode}: It stores the different kind of requests that a user can do  like manual irrigations, conditional irrigations, the obtainment of the sensor data.

\item \textbf{Device}: It stores the information of a device which corresponds to a user. It can have a default rule, so the user has not need to create its own. 

\item \textbf{Sensor}: It stores the value of a sensor of a specific type which correspond to a device.

\item \textbf{SensorType}: It stores the different type of sensors which are: Humidity, Temperature, Soil.

\item \textbf{Rule}: It stores all the configurations which correspond to a device. It can have RuleConditions and TimeConditions.

\item \textbf{RuleCondition}: It stores the conditions  with its respective value. It corresponds to a rule and a condition. If the sensors data of a device accomplish these conditions then an irrigation is produced.  

\item \textbf{RuleTimeCondition}: It corresponds to the conditions of a rule which a user decides when an irrigation has to be produced.  With the status we determine if this condition has to be considered.

\item \textbf{Condition}: It stores the different conditions which are: Temperature is lower than,Temperature is greater than, Humidity is lower than,  Humidity is greater than, Soil is lower than, Soil is greater than.
\end{itemize}

\section{Product backlog + Sprint backlog}
\begin{figure}[hbtp]
\centering
\includegraphics[scale=0.6]{figures/backlog.png}
\caption{Product backlog and sprint backlog issues}
\end{figure}

%% User stories
\begin{table}[htbp]
\centering
\begin{adjustbox}{angle=90,width=2.9in}
\begin{tabular}{|l|l|l|l|l|l|}
\hline
\textbf{Id} & \textbf{As a} & \textbf{I want to be able to} & \textbf{So that} & \textbf{Priority} & \textbf{Sprint} \\
\hline \hline
1 & User & Change the app language & I can visualize the content in some other languages & MEDIUM & 1 \\
\hline
2 & User & Visualize all the app screens & I can navigate throught all screens & HIGH & 1 \\
\hline
3 & User & Log in into the app & I can use the app functionalities & LOW & 2 \\
\hline
4 & User & Visualize the sensor characteristics of my plants & I can decide the best moment to irrigate & MEDIUM & 2 \\
\hline
5 & User & Activate or deactivate my devices & I can disconnect or connect my device & MEDIUM & 2 \\
\hline
6 & User & Create rules for each device & I can manage my plants automatically & MEDIUM & 2 \\
\hline
7 & User & Create conditions for each rule & I can establish more than one condition to activate the rule & MEDIUM & 2 \\
\hline
8 & Admin & View all users available & I can control the app access & LOW & 3 \\
\hline
9 & User & Irrigate manually from one device & I can irrigate my plants whenever i want & MEDIUM & 3 \\
\hline
10 & User & Visualize if my plants are infected & I can treat the infection as soon as possible & MEDIUM & 3 \\
\hline
11 & User & Irrigate automatically using rules and conditions & I can not worry about when to irrigate & MEDIUM & 3 \\
\hline
12 & User & Modify my device profile & I can change the image and the device name & LOW & 3 \\
\hline
13 & User & Update or delete the rules and conditions & I can modify my rules & MEDIUM & 3 \\
\hline
14 & Admin & Create, modify and delete users from the database & I can modify user into the database & MEDIUM & 4 \\
\hline
15 & Admin & Create, modify and delete devices associated with an user & I can modify devices into de database & MEDIUM & 4 \\
\hline
16 & Admin & Monitorize the cpu traffic and current device status & I can make user support (helpdesk) & LOW & 4 \\
\hline
\end{tabular}
\end{adjustbox}
\caption{Table of user stories}
\end{table}

\newpage

%% \section{Sprint backlog + dedication to each one}
%% Sprint backlog + dedication to each one

\section{Main screens desgin mockup}
%% Main screens desgin mockup
\begin{itemize}
\item Link: \url{https://marvelapp.com/5bj64b2/screen/62745623}
\end{itemize}

\section{Navigation schema between activities}
%% Navigation schema between activities
\begin{itemize}
\item Link: \url{https://drive.google.com/file/d/1ybN8-Wpv3C5P7Cu0UUbcai3s3eHySBIR/view}
\end{itemize}

\section{Economic viability}
\subsection{Who?}
The customer profile should be the one interested in controlling the irrigation and the health of the plants in your house in a comfortable, efficient and automated way. The population segment defined as target audience is broad and determined by these very simple conditions. \\

It can be focused both on private gardening and on a larger scale. Volume limits are set by the customer's business model or personal use. \\

Likewise, the company bets on the provision to distributors and garden shops, as well as the direct provision to individuals.\\

It will begin marketing the product in the area of Catalonia, where it will be made known to local distributors and will provide support in installation, configuration and maintenance to customers. 
\subsection{How many?}
An incremental and exponential progress of the user base is estimated as a function of the increase in the company's resources as a result of the successful marketing and expansion of the product. As such, it is estimated that some representative figures of success in the first year are: 

\begin{itemize}
\item To have marketed and installed and to have in maintenance 100 devices in particular. 
\item To have established supply agreements with 2 distributors. 
\end{itemize}
A significant potential progression would be:
\begin{table}[htbp]
\centering
\begin{tabular}{|l|l|l|l|l|l|}
\hline
\textbf{} & \textbf{Year 1} & \textbf{Year 2} & \textbf{Year 3} & \textbf{Year 4} & \textbf{Year 5} \\
\hline \hline
Particularities & 100 & 250 & 500 & 1000 & 2000 \\
\hline
Distributors & 2 & 4 & 6 & 8 & 10 \\
\hline
\end{tabular}
\caption{Table of potential marketing progression}
\end{table}

\subsection{Mareketing strategy}
The main comunication Garduino objectives are:
\begin{itemize}
\item Promote the notoriety of the brand and/or product marketed among the target audience.
\item Promote knowledge of product attributes, characteristics and performance, as well as competitive differentiation with respect to the same product market.
\item Establish and strengthen a connection between company and target audience, carrying out different strategies to familiarize and bring the user closer to the brand and product in marketing.
\item To constitute, with regard to the potential client, one of the alternatives of the market to contemplate when acquiring a product of the nature of the article in question.
\end{itemize}

The public of interest is understood to be the group of publics or audiences targeted by a marketing campaign. It should be noted that we can differentiate between primary and secondary audiences, depending on whether or not they are involved as potential end consumers.\\

As such, it has established a profile of primary interest audience of the characteristics described in the section "Who"?\\ 

The communication techniques used for the promotion and diffusion of brand and product will be the following:\\
\begin{itemize}
\item \textbf{Advertising in conventional media}: transmission of the advertising message through advertisements and programmed messages in media such as television, radio...
\item \textbf{Sales promotion}: alterations in the sales lot or in the price of the product in order to attract consumers.
\item \textbf{Fairs and exhibitions}: Participation and presence in fairs and exhibitions to make oneself known or to collaborate in the transmission of a message. 
\item \textbf{Sponsorship}: Association with other initiatives, businesses, causes and figures in order to carry out an exhibition of the brand in certain contexts, as well as to carry out an exploitation of the image associated with the activity in question, and to try to carry out a commercial exploitation of activities derived from the event.
\item \textbf{Personal sale}: Presential presentation and sale of the commercial to the client of the product in a physical space. This technique is usually carried out in commercial spaces, by means of informative and/or exhibition stands where the product is presented to the client and it is a question of negotiating a closing of sales with the same one. 
\item \textbf{Website}: Use of an informative web space where relevant company information can be announced, as well as support for online sales and subscription to other types of digital communications.
\item \textbf{Blogs}: Use of web spaces structured through a system of publication of articles and news over time, which can allow different degrees of participation by users, and through which it is intended to capture an interest in the user and redirect traffic to the main website or digital content that presents information on products that may be of interest to the potential consumer.
\item \textbf{Social networks}: Use of social network platforms to maintain a system of communication and promotion with the user, in addition to being a powerful tool for the approach and contact with it.
\item \textbf{E-mailing}: newsletter system that transmits information and advertising relevant to the brand and product through e-mail. 
\end{itemize}
\vspace{0.6in}
The communication model presented by Garduino is the multiphase model. This communication model not only involves commercials and experts or opinion leaders as influential and diffusion figures, but it will be the target public itself that will assume a role of active participation within the communicative process. These will act as "ambassadors" of a product, brand, good, service...\\ 

\newpage

This model manifests itself strongly in the growing digital and social network environments, and the search for the implementation of this kind of model by companies is evident in the incitement for the user to participate in one of the dissemination campaigns through these environments. It is also considered that this strategy will be based on the principles of inbound marketing:\\
\begin{figure}[hbtp]
\centering
\includegraphics[scale=1]{figures/marketing.png}
\caption{Inbound marketing scheme}
\end{figure}

\subsection{Monetization strategy}
The main sources of revenue for the business model will be: 
\begin{itemize}
\item Sale of the software to individuals
\item Sale of the software to distributors
\item Sale of the devices to private individuals
\item Sale of devices to suppliers
\item Installation service
\item Maintenance service
\item Special customization service
\end{itemize}

\subsection{App costs}
\begin{itemize}
\item \textbf{App type}: app for the management and monitoring of irrigation and pest control installations. 
\item \textbf{App platform}: Android (potentially IOS, in the future)
Coding: The start-up partners will act as developers of the project software. 
\item \textbf{Features}: Configuration of the installed devices, live monitoring, notification system for events such as detection of anomalies/pests...
\item \textbf{UI Design \& Graphics}: The interface design of the app must be intuitive and user-friendly, while covering a high degree of customization options for irrigation systems. 
\item \textbf{Marketing}: The stipulated communication campaign will be followed, whose investment will be proportional to the level of income generated at each moment of the life phase of the product. 
\item \textbf{Operational costs}: Costs for the Arduino parts of the prototypes, electricity costs and maintenance of the computers with which the project is programmed. 
\end{itemize}
\end{document}